%================Thesis Abstract===============
\begin{center}
\Large{\textbf{ABSTRACT}}\\
\end{center}
%=============================================
Intrusion Detection Systems have become an essential part of computer network security. It acts as a first-line defense from cyber-attacks by analyzing the various attributes of a data packet and identifying it as a malicious, or an ordinary one. IDSs have been in use for decades in computer security, however, early implementations of IDSs could only detect attacks that were well known because of the knowledge based model \cite{ids_taxonomy}. With the advent of Machine Learning, new possibilites have opened up for IDSs to detect novel attacks by analyzing the behavior of data packets in a network. Such systems have come a long way and in the present time, promise acceptable performance. However, the algorithms that run at the core of these systems are computationally expensive, making them fairly accurate, but almost unimplementable in very low powered devices, such as nodes of a WSAN, or in an IoT based setup. Devices in these environments have very low computational power as they are designed to provide very minimal functionality and deployed in large numbers to create a dense network. Reducing the additional overhead and power consumption in such systems is an important topic of research. This work attempts to rank various machine learning and data mining techniques on the basis of their accuracy and time required to train and classify network data.

% Refer to ACM Computing Classification System Taxonomy, read first page and following the instruction given therein.=============
\par
\textbf{[Security and Privacy]}: Intrusion/Anomaly Detection and Malware Mitigation--Intrusion detection systems
