
\chapter{Introduction}
\label{ch:INTR}
\section{Overview}

\paragraph{}
An Intrusion Detection System (IDS) is an application that can be installed on a system or a network. The application may be a piece of software, or a device that can be plugged into the interface as a module. The IDS, once plugged into the system, scans for activity, and based on various features pertaining to the activity, classifies it as normal, or malicious. It must be noted that the IDS itself does not provide any access control, and is only responsible for the \textit{detection} of an attack. As a result, IDSs must be bundled with other tools to prevent an attack. Typically, all detected malicious activity is reported to a Security Information and Event Management (SIEM) system, which combines reports from multiple sources and uses alarm filtering techniques to distinguish malicious activity from false alarms.

\paragraph{}
Many attempts have been made to classify various kinds of IDSs. According to \cite{ids_taxonomy}, on the basis of method of detection, two types of IDSs have been identified:
\begin{itemize}
    \item \textbf{Knowledge Based}: Knowledge Based IDSs accumulate knowledge about attacks and look for similar data to detect an attack. Since these IDSs look for very specific signatures, they provide very accurate detection of common attacks. However, in a case where an attack has a signature that has never been seen before, this kind of system fails. This type of IDS is also called \textbf{misuse detection} IDS.

    \item \textbf{Behavior Based}: Behavior Based IDSs work with the assumption that attacks can be detected if the behavior of the system or the users deviate from the normal expected pattern. Many sources are combined to gather enough data to teach a system what normal behavior looks like and to create a model. The IDS then compares the current activity with this model and reports if it detects an anomaly. These IDSs are also called \textbf{anomaly detection} IDSs. Since these IDSs are based on predictive models, they tend to have lower accuracy, and higher false alarm rate when compared to a knowledge based IDS. However, they are effective against novel attacks as any kind of attack is detected as an anomaly, or a deviation from the norm.
\end{itemize}

\paragraph{}
A third type of IDS is also often seen in literature, called the \textbf{hybrid detection} IDS. This is a combination of the two types. Typically, an activity is first passed through a knowledge based system. If the first filter claims the activity to be an attack, the system terminates and the activity is reported. If the activity passes the first filter, it is passed through a behavior based system, where even if the signature could not be found in the dictionary of attacks, an anomaly is detected and the activity is reported. The signature of this activity is then recorded in the knowledge based system for future reference. An activity is not reported as malicious only if it passes both of the filters. Although the process of identification is often sped up compared to the other two types of IDSs, the accuracy is bottle-necked by the behavior based system used. Running such a system with a poor behavior based system also possess the risk of saving false signatures in the initial filter. If an activity passes through the first filter, and the second filter raises a false alarm, the signature of the (in reality) normal activity is recorded as malicious, and is reported in subsequent instances of similar activities.

\paragraph{}
Thus, choosing the right behavior based model becomes an important part of building an IDS. Knowledge based systems are not future proof, and it is often too late for a system or a network to be affected by an attacker even once, regardless of how well the IDS reports subsequent activities of the same. Hybrid systems may provide quick classification, but are bottle-necked by the high false positive rate of the behavior based model used.

\section{Objectives}

\paragraph{}
The broad objective of this study was to analyze different machine learning and data mining models on the basis of their accuracy as well as execution time.\\
Specifically, the objectives can be stated as follows:
\begin{itemize}
    \item To determine useful features in classifying network data.
    \item To find appropriate measures to score the different models.
    \item To study the causes of different models performing differently.
    \item To open scope for future studies to improve the given score.
\end{itemize}

\section{Limitations of Study}

\begin{itemize}
    \item The study has been conducted with a static database for training and testing. Although the dataset has been curated fairly recently (2015), it is at best, a rough approximate of live data flowing through a network.
    \item Many complex algorithms (such as deep neural networks, multi-layer perceptron etc.) which have proved to be very accurate have not been tested simply because the execution time of these algorithms were too high for it to fetch a comparable score.
    \item Although \cite{mlp_17} suggests a faster and efficient way to implement a multi-layer perceptron with binary weights, it has also claimed a reduction of accuracy by up to 4 times for the same, and hence has not been considered in the comparison.
\end{itemize}
